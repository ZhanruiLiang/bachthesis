\chapter{Summary}

\label{Chapter:Summary}
\lhead{\emph{Summary}}

In the previous chapters, we described the most significant works in our application.

The first problem we solved is
rendering of sheet music. Scores are read from MusicXML file, which provide the necessary information for rendering.
However, not all information is given directly. Some information are context relevant and some need to be inferred by
conventional rules. 
The most challenging part of the rendering problem is to determinate the slope and stacking for beams. It is solved b
y sorting and iterating techniques. At last we got quite pleasing results.

Having audio playback in musical applications is always helpful.
The second problem we solved is playing sounds. The most important step is to generate the note sequence from the score.
The generated note sequence is also used to solve the fingering arrangement problem. 

The next problem we solved is the fingering arrangement problem. We proposed a expressive model and than used dynamic programming 
to solve the problem. The most challenging part of this method is to calculate the cost functions. We set up essential rules to 
formulate the calculation of cost functions. In the cost model we have several parameters that were unknown at the beginning. They 
are found out by experiment later. The results we got are usually properly arranged and playable. However, they are not always the  
same with the published version. It is due to that the rules in our cost model can not reflect some conventions.

To further improve the work, we should add tie note and special playing techniques to the model.
Special techniques such as hitting, picking and sliding may affect the fingering arrangement. Taking them into our model can improve
the fingering arrangement result.
