\chapter{Introduction}

\label{Chapter:Introduction}
\lhead{\emph{Introduction}}

\section{Background}
Guitar is a popular musical instrument that can be used perform a variety of musical genres.
Typically, a guitar has 6 strings and 17 or more frets, dividing about 100 positions on it's fretboard.
When each position is pressed by the player's left hand and the corresponding string is plucked by the right hand, a sound is produced.
Without special techniques, each position played by this manner will produce a sound with specific pitch.
Different from the piano, which has a one-to-one correspondence between each pitch and each key on the keyboard, a sound with specific pitch can be produced on different positions on the guitar fretboard.
From another point of view, a guitar can produce about only 50 level of pitches while it has about 100 positions, there must be some positions producing same pitch, though there may be difference in the sound effect.
Therefore, we need to choose some proper positions for the notes that we want to play.

Typical musical scores is written on the five line staff, from which we can know each note's pitch easily.
However, few scores has fingering information. It's somewhat difficult for beginners to determine the proper position for each note on the score.

Tablature is a form of musical score that solve this problem to some extent.
It's widely used for string instruments.
It adopt a different format to denote music, by telling which position(both string and fret) to use for each note, instead of telling the note's pitch.
However, tablature is not always available, since many guitar pieces have only the score version, especially pieces for classical guitar.
As \citep{constructing-system} mentioned, one reason for this is that once a guitar player become experienced enough, he/she don't need it that much.

\section{Previous Works}
% TODO
\section{Objectives}
% TODO
\section{Contribution}
% TODO
\section{Structure of this Thesis}
This thesis is divided into 3 main chapters.
% TODO
% Chapter 2 first introduces some basic concepts about sheet music and then discusses the processing of sheet music, including parsing, rendering and sound playing.
% Chapter 3 presents the fingering arrangement problem.
